%No rights reserved under CC0 license.
%See file LICENSE in git root for further details.

\documentclass[11pt,twoside]{article}
\usepackage[english]{babel}
\usepackage{cite}
\usepackage{url}
\usepackage[a4paper, total={6in, 8in}]{geometry}
\usepackage{fancyhdr}
\usepackage{parskip}
\usepackage{multicol}

\usepackage{tikz}
\usepackage[siunitx]{circuitikz}
\usepackage{amsmath}

\begin{document}
	\tableofcontents
	\section{The Menstrual Cycle}
		The menstrual cycle is controlled by hormones. These are made both in the brain and in the reproductive organs.
		
		The \emph{endometrium} (lining of the uterus) is thickest right before menstruation.
		\subsection{Hormones}
			\begin{itemize}
				\item{FSH \emph{(Follicle stimulating hormone)} triggers estrogen. It stimulates the follicles to develop}
				\item{Estrogen triggers and spikes LH. It promotes \emph{endometrium repair} and \emph{follicle bursts}}
				\item{LH \emph{(Luteinizing hormone)} triggers progesterone excretion from the \emph{corpeus luteum (gulkropp)}. It also stimulates \emph{ovulation}}
				\item{Progesterone triggers FSH and decreases LH. It stimulates the \emph{endometrium} to get thicker and, if the egg is not fertilized, breaks down the \emph{corpus luteum}, triggering menstruation}
			\end{itemize}
		\section{Embryogenesis}
			\subsection{Germ Layers}
				\begin{itemize}
					\item{\emph{Ectoderm} becomes brain and nervous system}
					\item{\emph{Endoderm} becomes digestive system and respiratory system}
					\item{\emph{Mesoderm} becomes musculoskeletal system, heart, vascular system, and circulatory system}
				\end{itemize}
\end{document}
