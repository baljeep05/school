\documentclass[11pt,twoside]{article}
\usepackage[english]{babel}
\usepackage{cite}
\usepackage{url}
\usepackage[a4paper, total={6in, 8in}]{geometry}
\usepackage{fancyhdr}
\usepackage{parskip}

\usepackage{tikz}
\usepackage[siunitx]{circuitikz}
\usepackage{amsmath}

\begin{document}
	\tableofcontents	
	\section{Review circuits}
		A \emph{circuit} is a loop of \emph{conductive material}
		
		\begin{circuitikz}
			\draw (0,0)
			to[short] (0,1)
			to[short] (1,1)
			to[short] (1,0)
			to[short] (0,0);
		\end{circuitikz}
		\\
		An example of a very boring circuit
		
		\begin{circuitikz}
			\draw (0,0)
			to[battery1=U] (0,1) % The voltage source
			to[short] (1,1)
			to[short] (1,0)
			to[short] (0,0);
		\end{circuitikz}
		\\
		A slightly less boring circuit
		
		\begin{circuitikz}
			\draw (0,0)
			to[battery1=U] (0,2) % The voltage source
			to[short] (1,2)
			to[resistor] (1,0)
			to[short] (0,0);
		\end{circuitikz}
		\\
		A circuit with a battery and a resistor
		
		\begin{circuitikz}
			\draw (0,0)
			to[battery1=] (1,0); % The voltage source
		\end{circuitikz}
		\\
		The longer leg of the battery is the negative pole, while the shorter one is the positive pole.
		
		\subsection{Ohm's Law}
			\emph{Ohm's Law} states that $U = I \times R$ where $U$ is voltage, measured in volts ($V$), $I$ is current, measured in amps ($A$), and $R$ is resistance, measured in ohms ($\Omega$).
		\newpage
		\subsection{Kirchoff's Law}
			\emph{Kirchoff's Law} states that, at a junction, currents before and after must equal.
			
			\begin{circuitikz}
				\draw (0,0)
				to[short=I] (1,0)
				to[short] (1,0.5)
				to[short=${I_{2}}$] (2,0.5)
				to[short] (1,0.5)
				to[short] (1,-0.5)
				to[short=${I_{1}}$] (2,-0.5);
			\end{circuitikz}
			\\
			In this case, $I = I_1 + I_2$
	\section{Multiple resistors}
		\subsection{Paralell}
			In Paralell, the voltage is the same for all components, but current and resistance may change. In a circuit that looks like this, $\frac{1}{R} = \frac{1}{R_1} + \frac{1}{R_2}$ \\
			\begin{circuitikz}
			\draw (0,0)
				to[battery1=$U$ $I$] (2,0) % The voltage source
				to[short] (2,-3)
				to[resistor=$R_1$ $I_2$] (0,-3)
				to[open] (2,-1.5)
				to[resistor=$R_2$ $I_2$] (0,-1.5)
				to[open] (0,-3)
				to[short] (0,0);
			\end{circuitikz}
			\\
			The following is a proof of this:
			\begin{center}
			According to Ohm's Law,
			\[U = I \times R \implies I = \frac{U}{R}\]  
			\[U = I_1 \times R_1 \implies I_1 = \frac{U}{R_1}\] 
			\[U = I_2 \times R_2 \implies I_2 = \frac{U}{R_2}\] 
			According to Kirchoff's Law ($I = I_1 + I_2$)
			\[\frac{U}{R} = \frac{U}{R_2} + \frac{U}{R_2}\]
			\[\frac{1}{R} = \frac{1}{R_2} + \frac{1}{R_2}\]
			\end{center}
			This shows that adding a resistor in paralell increases $R^{-1}$, which means that $R$ gets smaller.
		\subsection{Series}
			In series, the current is the same for all components, but voltage and resistance may change. In a circuit that looks like this, $R = R_1 + R_2$. \\
			\begin{circuitikz}
			\draw (0,0)
				to[battery1=U] (4,0) % The voltage source
				to[short] (4,-1)
				to[resistor=$R_1$] (2,-1)
				to[resistor=$R_2$] (0,-1)
				to[short] (0,0);
			\end{circuitikz}
			\\
			The following is a proof of this, using Ohm's law.
			\[U_1 = I \times R_1\]
			\begin{center}
			and
			\end{center}
			\[U_2 = I \times R_2\]
			\begin{center}
			and
			\end{center}
			\[U_1 + U_2 = U\]
			\begin{center}
			using Ohm's Law  we can now find the resistances by substituting $U$ for $I \times R$
			\end{center}
			\[I \times R = I \times R_1 + I \times R_2\]
			\begin{center}
			the currents cancel.
			\end{center}
			\[R = R_1 + R_2\]
	\section{Measuring devices}
		\subsection{Voltmeters}
			 Components in paralell have the same voltage accross, thus voltage meters should be connected in paralell with the resistor.
		\subsection{Current meters}
			Components in series have the same current flowing through them, thus current meters should be connected in series with the resistor.
	\section{Multiple Batteries}
		\subsection{Series}
			If there are batteries in series, the voltage adds and current remains the same.
		\subsection{Paralell}
			If the batteries are in paralell, the current adds and voltage remains the same. This means the batteries will last for longer, or be able to output more current.
\end{document}
