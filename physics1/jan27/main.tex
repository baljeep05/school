%No rights reserved under CC0 license.
%See file LICENSE in git root for further details.

\documentclass[11pt,twoside]{article}
\usepackage[english]{babel}
\usepackage{cite}
\usepackage{url}
\usepackage[a4paper, total={6in, 8in}]{geometry}
\usepackage{fancyhdr}
\usepackage{parskip}
\usepackage{multicol}

\usepackage{tikz}
\usepackage[siunitx]{circuitikz}
\usepackage{chemfig}
\usepackage{amsmath}
\usepackage[version=4]{mhchem}

\begin{document}
	\setchemfig{atom sep=15pt}
	\section{Electric Potential}
		
		Electric potential is defined as \[ V = \frac{\Delta E_p}{Q} \] where $V$ is measured in volts.
			
		\subsection{"Potential Walks"}
		
			The potential walk is a method for finding the electric potential in a given spot on a circuit.
			
			\subsubsection*{Passing a Battery from Positive to Negative}
				When passing a voltage source from positive to negative, subtract $u$ from $v$
				
			\subsubsection*{Passing a Battery from Negative to Positive}
				When passing a battery voltage source from negative to positive, add $u$ to $v$
				
			\subsubsection*{Passing a Resistor from Positive to Negative}
				When passing a resistor (or other component) in the direction of the current, we subtract $I \times R$ from $v$
				
			\subsubsection*{Passing a Resistor from Negative to Positive}
				When passing a resistor (or other component) against the direction of the current, we add $I \times R$ to $v$
				
		\subsection{Grounding}
			A \emph{ground point} is a point where the circuit connects to the ground\footnote{Ground is usually defined as the earth}. At a ground point $E_p = 0$. 
			
			\begin{circuitikz}
				\draw (0,0)
				to[battery1={U=1.5v}] (0,2) % The voltage source
				to[short] (1,2)
				to[resistor={$R = 5 \Omega$}] (1,0)
				to[short] (0,0);
				\draw (0,0) to (0,0) node[ground]{}; 
			\end{circuitikz}
\end{document}
