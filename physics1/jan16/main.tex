%No rights reserved under CC0 license.
%See file LICENSE in git root for further details.

\documentclass[11pt,twoside]{article}
\usepackage[english]{babel}
\usepackage{cite}
\usepackage{url}
\usepackage[a4paper, total={6in, 8in}]{geometry}
\usepackage{fancyhdr}
\usepackage{parskip}
\usepackage{multicol}

\usepackage{tikz}
\usepackage[siunitx]{circuitikz}
\usepackage{chemfig}
\usepackage{amsmath}
\usepackage[version=4]{mhchem}

\begin{document}
	\setchemfig{atom sep=15pt}
	\section{EMF \& Terminal Voltage in Batteries}
		Thus far batteries have been assumed to be "ideal" power sources, i.e. a perfect source of voltage, with no resistance.
		
		This is incorrect, as batteries have an internal resistance. The battery consists of a volatge source called the \emph{Electromotive Force} (EMF). It is written as $ \varepsilon $. It is measured in Volts ($V$) The battery also has an internal resistance, $R_i$. It is measured in Ohms ($\Omega$)
		
		\subsection{Terminal Voltage}
			The \emph{terminal voltage} is the volatge between the terminals (poles) on a battery with an open circuit. It is the same as $\varepsilon$
			
			\subsubsection{The Short-circuit current}
				The \emph{short-circuit current} is the current that would flow through a completed circuit with no components on it. It can be found by solving for $I$ in the following equation: $ \varepsilon = R_i \times I $.
				
			\subsubsection{U-I diagram}
				The U-I diagram is a diagram with voltage ($U$) on the y-axis and current ($I$) on the x-axis. The equation of the line on the diagram can be written as $U = \varepsilon - R_I \times I$
				
		\subsection{Ohm's 2nd Law}
			Ohm's 2nd law states that $\varepsilon = (R_i + R_y) I $ where $R_i$ is the internal resistance, $R_y$ is the external resistance\footnote{Resistance outside the battery}, $I$ is the current, and $\varepsilon$ is the \emph{EMF}
\end{document}
