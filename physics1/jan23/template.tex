%No rights reserved under CC0 license.
%See file LICENSE in git root for further details.

\documentclass[11pt,twoside]{article}
\usepackage[english]{babel}
\usepackage{cite}
\usepackage{url}
\usepackage[a4paper, total={6in, 8in}]{geometry}
\usepackage{fancyhdr}
\usepackage{parskip}
\usepackage{multicol}

\usepackage{tikz}
\usepackage[siunitx]{circuitikz}
\usepackage{chemfig}
\usepackage{amsmath}
\usepackage[version=4]{mhchem}

\begin{document}
	\setchemfig{atom sep=15pt}
	\tableofcontents
	\section{Electric Charge and Force}
		Charge is fundamentally a quantity related to an object. A proton has a positive charge while the electron has a negative charge. We can use charge to create an \emph{attractive force} or \emph{repulsive force}, depending on the sign of the charges involved. If the charges have different signs, the forces will attract. If they are different, they will repel. The force is called the \emph{coulomb force, electric force,} or \emph{electrostatic force}.
		
		Charge is $Q$ or $q$, and is measured in the unit Coulomb (C). Coulomb is defined as $\frac{A}{S}$ 
		
		An object with 0 coulombs of charge is called \emph{Neutral}
		
		\subsection{Coulomb's Law}
			Given two forces, $Q_1$ and $Q_2$, and the distance $r$ (given in metres), the \emph{coulomb force} can be calculated with the following equation:
			\[ F = k \frac{Q_1 \times Q_2}{r^2} \]
			Where $k$ is a constant at $8.99 \times 10^9 N \frac{m^2}{C^2}$
			
			If $F < 0$, the charges will attract. \\ If $F > 0$, the charges will repel.
			
			\subsubsection{A new Foe has Appeared!}
				If we add more charges, let's say, $Q_3$, we must calculate the forces between all particles, in this case, $Q_1$ and $Q_2$, $Q_1$ and $Q_3$, and  $Q_2$ and $Q_3$.
				
	\section{Elementary Charge}
		An object cannot have any charge, only integer multiples of the \emph{elementary charge}, $e$. $e = 1.60 \times 10^{-19} C$.
		
		The electron ($e^-$) has a charge of $-e$ while the proton has a charge of $+e$
		
		In a wire conducting 1A, there are $\frac{1}{e}$ or approximately $6 \times 10^{18}$ electrons passing every second.  
\end{document} 
