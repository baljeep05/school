%No rights reserved under CC0 license.
%See file LICENSE in git root for further details.

\documentclass[11pt,twoside]{article}
\usepackage[english]{babel}
\usepackage{cite}
\usepackage{url}
\usepackage[a4paper, total={6in, 8in}]{geometry}
\usepackage{fancyhdr}
\usepackage{parskip}
\usepackage{multicol}

\usepackage{tikz}
\usepackage[siunitx]{circuitikz}
\usepackage{chemfig}
\usepackage{amsmath}
\usepackage[version=4]{mhchem}

\begin{document}
	\setchemfig{atom sep=15pt}
	\tableofcontents
	
	\section{Alkenes}
		Alkenes are hydrocarbons with one or more double bonds.
		\\ \\
		\chemfig{C(-[:0]C(=[:0]C(-[:0]H)(-[:270]H))(-[:270]H))(-[:90]H)(-[:180]H)(-[:270]H)}
		\\ \\
		This is an example of an alkene, propene.
		\subsection{Naming Alkenes}
			Naming alkenes is very similar to naming alkanes. You count the carbons the same way you do in alkanes with substituent groups, giving priority to having low numbers for \emph{functional groups}\footnote{The double bond, in this case.}, and then name the compound according to the following: Alk-x-zene where "Alk" is the first part of the name of the corresponding alkane, "x" is the location(s) of functional groups, and "z" is the prefix for the number of functional groups (di, tri, tetra, etc.) if there is only one functional group there is no prefix. Substituent groups are named in the same way as for alkanes.
			
			For example, a pentene with double bonds at carbon number 1 \& 2 would be pent-1,2-diene. 
			
	\section{Alkynes}
		Alkynes are hydrocarbons with one or more triple bonds.
		\\
		\chemfig{C(-[:0]C([:0]C(-[:0]H)))(-[:90]H)(-[:180]H)(-[:270]H)}
		\\
		This is an example of an alkyne, propyne.
		\subsection{Naming Alkynes}
			Naming alkynes is the same as naming alkenes, but with \emph{-yne} instead of \emph{ene} The formula for the name is therefore Alk-x-zyne. Substituent groups are added to the name in the same way as previously.
			
			For example, a hexyne with triple bonds at carbon number 1 \& 3 \& 6 would be hex-1,3,6-triyne
\end{document}
