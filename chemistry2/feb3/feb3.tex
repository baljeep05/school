%No rights reserved under CC0 license.
%See file LICENSE in git root for further details.

\documentclass[11pt,twoside]{article}
\usepackage[english]{babel}
\usepackage{cite}
\usepackage{url}
\usepackage[a4paper, total={6in, 8in}]{geometry}
\usepackage{fancyhdr}
\usepackage{parskip}
\usepackage{multicol}

\usepackage{tikz}
\usepackage[siunitx]{circuitikz}
\usepackage{chemfig}
\usepackage{amsmath}
\usepackage[version=4]{mhchem}

\begin{document}
	\setchemfig{atom sep=15pt}
	
	In a reaction between organic molecules, it is \emph{only} the functional groups that participate in the reaction.
	
	\textbf{Example:}
		\schemestart
			\chemname{\chemfig{R-C(-[:-30]OH)=[:30]O}}{Carboxylic acid} 
			\+
			\chemname{\chemfig{R’OH}}{Alcohol} 
			\arrow{->} 
			\chemname{\chemfig{R-C(-[:-30]OR’)=[:30]O}}{Ester} 
			\+
			\chemname{\chemfig{H_2O}}{Water} 
		\schemestop
	
	In this reaction, the hydroxide from the alcohol and the hydrogen from the acid will react, forming water. This will leave the alkyl with a free radical from the alcohol to bond to the \ce{O2CR} from the carboxylic acid. This yields an ester. 
	
	\section{Reaction Mechanisms of Alkanes}
		There are two ways in which an alkane can react, 
		\begin{enumerate}
			\item{Combustion}
			\item{Free radical substitution}
		\end{enumerate}
		
		\subsection{Combustion}
			There are two types of alkane combustion, complete and incomplete. In combustion, all bonds are broken and then reformed.
			\subsubsection{Complete Combustion}
				Complete combustion requires excess air/oxygen.
				
				\ce{R + O2 -> H2O + CO2}
				
				A specific example would be:
				
				\schemestart
					\chemname{\ce{C2H6}}{Ethane}
					\+
					\chemname{\ce{7O2}}{Oxygen}
					\arrow{->}
					\chemname{\ce{6H2O}}{Water}
					\+
					\chemname{\ce{4CO2}}{Carbon Dioxide}
				\schemestop
			\subsubsection{Incomplete Combustion}
				Incomplete combustion is when air/oxygen is not in excess.
				
				\ce{R + O2 -> H2O + CO}
				
				and
				
				\ce{R + O2 -> H2O + C}
				
				A more specific example would be the incomplete combustion of ethane:
				
				\ce{C2H6 + 5O2 -> 6H2O + 4CO}
				
				This reaction yields carbon monoxide, a very dangerous gas. The second reaction in the incomplete combustion of ethane is:
				
				\ce{C2H6 + 3O2 -> 6H2O + 4C}
				
				This reaction yields carbon, or soot.
	
		\subsection{Free Radical Substitution}
			\schemestart
					Alkane
					\+
					Halogen
					\arrow{->[UV][Light]}
					Halogenoalkane
					\+
					Hydrochloric Acid
				\schemestop
				\subsubsection{Mechanism of Reaction}
					\textbf{Initiation}
						The halogen is broken apart by homolytic fission, yielding two free radicals.
						\schemestart
							\setchemfig{bond offset=5pt}
							\chemfig{Cl-[:0]Cl}
							\arrow{->[UV][Light]}
							\ce{Cl. + Cl.}
						\schemestop
					
					\textbf{Propogation}
						The halogen free radicals react with the other compounds present in the reaction mixture in some of the following ways:
						
						\ce{CH4 + Cl. -> CH3. + HCl} \\
						\ce{CH3. + Cl2 -> CH3Cl + Cl.} \\
						\ce{Cl. + CH3Cl -> CHCl2. + HCl} \\
						\ce{CH2Cl. + Cl2 -> CH2Cl2 + Cl.} \\
						\ce{Cl. + CH2Cl2 -> CHCl2. + HCl} \\
						\ce{CHCl2. + Cl2 -> CHCl3 + Cl.} \\
						\ce{Cl. + CHCl3 -> CCl3. + HCl} \\
						\ce{CCl3. + Cl2 -> CCl4 + Cl.}\footnote{\ce{CCl4} is very very bad for you}
						
					\textbf{Termination}
						Radicals combine to create stable compounds, ending the chain reaction.
						
						\ce{Cl. + CH3. -> CH3Cl} \\
						\ce{Cl. + Cl. -> Cl2} \\
						\ce{CH3. +  CH3. -> C2H6}
						
						The reaction also has the byproducts of all the propogation steps.
						
\end{document}

