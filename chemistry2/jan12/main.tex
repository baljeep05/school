\documentclass[11pt,twoside]{article}
\usepackage[english]{babel}
\usepackage{cite}
\usepackage{url}
\usepackage[a4paper, total={6in, 8in}]{geometry}
\usepackage{fancyhdr}
\usepackage{parskip}

\usepackage{tikz}
\usepackage{chemfig}
\usepackage{amsmath}
\usepackage{mhchem}

\begin{document}
	\tableofcontents
	\section{Chapter Goals}
		\subsection{Part 1}
			\begin{itemize}
				\item{Name the molecules according to \emph{IUPAC}}
				\item{Identify the \emph{functional Groups} in a \emph{homologous series}}
			\end{itemize}
		\subsection{Part 2}
			\begin{itemize}
				\item{Chemical reactions}
				\item{Reaction mechanisms}
			\end{itemize}
	\section{Hydrocarbons}
	\emph{Hydrocarbons} are molecules made of only carbon and hydrogen atoms covalently bonded to each other. All bonds are nonpolar\footnote{Difference of electronegativities less than 0.5} and covalent. The carbon atoms \emph{always} have 4 bonds.
		\subsection{The Functional Group}
			The \emph{functional group} gives the molecule a specific property. It can consist of one or more atoms. It is the only part of the molecule that parttakes in an organic reaction.
		\subsection{Homologous Series}
			A \emph{homologous series} is a group of molecules that share a \emph{functional group} but have different numbers of carbons in the \emph{carbon chain}
		\subsection{Alkanes}
			The \emph{alkanes} are molecules made entirely of carbon and hydrogen joined by single covalent bonds. The \emph{-ane} suffix indicates an alkane. The general formula of an alkane is $C_n H_{2n+2}$
			\subsubsection{List of Alkanes}
				\begin{enumerate}
					\item{Methane}
					\item{Ethane}
					\item{Propane}
					\item{Butane}
					\item{Pentane}
					\item{Hexane}
					\item{Heptane}
					\item{Octane}
				\end{enumerate}
\end{document}
