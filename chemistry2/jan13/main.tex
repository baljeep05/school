%No rights reserved under CC0 license.
%See file LICENSE in git root for further details.

\documentclass[11pt,twoside]{article}
\usepackage[english]{babel}
\usepackage{cite}
\usepackage{url}
\usepackage[a4paper, total={6in, 8in}]{geometry}
\usepackage{fancyhdr}
\usepackage{parskip}
\usepackage{multicol}

\usepackage{tikz}
\usepackage[siunitx]{circuitikz}
\usepackage{chemfig}
\usepackage{amsmath}
\usepackage[version=4]{mhchem}
\setchemfig{atom sep=15pt}

\begin{document}
	\tableofcontents

	\section{Hydrocarbons}
		\subsection{Properties of Hydrocarbons}
			Melting and boiling point increase as the carbon chain gets longer. Melting and boiling points can only be compared between molecules in the same homologous series. \textbf{This is because different homologous series may have different intermolecular bonds than others}.	
			
		\subsection{Alkanes}
			\subsubsection{Different Ways of Representing the Alkanes}
				{\setchemfig{atom sep=12pt}
				\begin{tabular}{|c|c|c|c|} \hline
					Name & Molecular formula & Full Structural Formula & Condensed Formula \\ \hline
					Methane & $\ce{CH4}$ & \chemfig{C(-[:0]H)(-[:90]H)(-[:180]H)(-[:270]H)} & $\ce{CH4}$ \\ \hline
					Ethane & $\ce{C2H6}$ & \chemfig{C(-[:0]C(-[:0]H)(-[:90]H)(-[:270]H))(-[:90]H)(-[:180]H)(-[:270]H)} & $\ce{CH3CH3}$ \\ \hline
					Propane & $\ce{C3H8}$ & \chemfig{C(-[:0]C(-[:0]C(-[:90]H)(-[:0]H)(-[:270]H))(-[:90]H)(-[:270]H))(-[:90]H)(-[:180]H)(-[:270]H)} & $\ce{CH3CH2CH3}$ \\ \hline
					Butane & $\ce{C4H10}$ & \chemfig{C(-[:0]C(-[:90]H)(-[:270]H)(-[:0]C(-[:90]H)(-[:270]H)(-[:0]C(-[:90]H)(-[:270]H)(-[:0]H))))(-[:180]H)(-[:90]H)(-[:270]H)} & $\ce{CH3CH2CH2CH3}$ \\ \hline
					Pentane & $\ce{C5H12}$ & \chemfig{C(-[:90]H)(-[:180]H)(-[:270]H)(-[:0]C(-[:90]H)(-[:270]H)(-[:0]C(-[:90]H)(-[:270]H)(-[:0]C(-[:90]H)(-[:270]H)(-[:0]C(-[:90]H)(-[:0]H)(-[:270]H)))))} & $\ce{CH3CH2CH2CH2CH3}$ \\ \hline
					Hexane & $\ce{C6H14}$ & \chemfig{C(-[:90]H)(-[:180]H)(-[:270]H)(-[:0]C(-[:90]H)(-[:270]H)(-[:0]C(-[:90]H)(-[:270]H)(-[:0]C(-[:90]H)(-[:270]H)(-[:0]C(-[:90]H)(-[:270]H)(-[:0]C(-[:90]H)(-[:270]H)(-[:0]H))))))} & $\ce{CH3CH2CH2CH2CH2CH3}$ \\ \hline
					Heptane & $\ce{C7H16}$ & \chemfig{C(-[:90]H)(-[:180]H)(-[:270]H)((-[:0]C(-[:90]H)(-[:270]H)((-[:0]C(-[:90]H)(-[:270]H)((-[:0]C(-[:90]H)(-[:270]H)((-[:0]C(-[:90]H)(-[:270]H)((-[:0]C(-[:90]H)(-[:270]H)((-[:0]C(-[:90]H)(-[:270]H)(-[:0]H)))))))))))))} & $\ce{CH3CH2CH2CH2CH2CH2CH3}$ \\ \hline
					Octane & $\ce{C8H18}$ & \chemfig{C(-[:90]H)(-[:180]H)(-[:270]H)((-[:0]C(-[:90]H)(-[:270]H)((-[:0]C(-[:90]H)(-[:270]H)((-[:0]C(-[:90]H)(-[:270]H)((-[:0]C(-[:90]H)(-[:270]H)((-[:0]C(-[:90]H)(-[:270]H)((-[:0]C(-[:90]H)(-[:270]H)(-[:0]C(-[:90]H)(-[:270]H)(-[:0]H))))))))))))))} & $\ce{CH3CH2CH2CH2CH2CH2CH2CH3}$ \\ \hline
				\end{tabular}
				}
				\subsubsection{Skeletal Formulae}
					\begin{itemize}
						\item{Propane \chemfig{-[1]-[-1]}}
						\item{Butane \chemfig{-[1]-[-1]-[1]}}
						\item{Pentane \chemfig{-[1]-[-1]-[1]-[-1]}}
						\item{Hexane \chemfig{-[1]-[-1]-[1]-[-1]-[1]}}
						\item{Heptane \chemfig{-[1]-[-1]-[1]-[-1]-[1]-[-1]}}
						\item{Octane \chemfig{-[1]-[-1]-[1]-[-1]-[1]-[-1]-[1]}}
					\end{itemize}
				\subsubsection{Cyclic Carbon Chain}
					Cyclopentane ($\ce{C5H10}$) \\ \\
					{\scriptsize \chemfig{*5(-----)}} \\ \\
					Alkanes can also occur as circular molecules. In this case their general formula is $\ce{CnH2n}$

	\section{IUPAC Naming Conventions}
		\subsection{The Rules}
			\begin{enumerate}
				\item{Find the longest continuous carbon chain. This will be the base of the name.}
				\item{Check if there are any groups of atoms/atoms that are not part of the longest carbon chain. These groups are called \emph{substituent groups}.}
				\item{Number the individual carbon atoms on the longest carbon chain so that the substituent groups are attached to the lowest number possible.}
				\item{If there is more than one substituent group, they must be sorted in alphabetical order.}
				\item{If there are two identical substituent groups, add the prefix \emph{di} for 2, \emph{tri} for 3, \emph{tetra} for 4, \emph{penta} for 5, and so on.}
			\end{enumerate}
		\subsection{Examples}
			\begin{enumerate}
				\item{Name this molecule: \\ \\ {\scriptsize \chemfig{C(-[:0]C(-[:0]C(-[:90]H)(-[:0]H)(-[:270]H))(-[:90]\ce{CH2})(-[:270]H))(-[:90]H)(-[:180]H)(-[:270]H)}} \\ \\ \emph{Answer:} 2methyl-propane, as there is a methyl group on the 2nd carbon of the propane.}
				\item{Name this molecule: \\ \\ {\scriptsize \chemfig{C(-[:90]H)(-[:180]H)(-[:270]H)(-[:0]C(-[:90]H)(-[:270]H)(-[:0]C(-[:90]H)(-[:270]H)(-[:0]C(-[:90](-[:90]C(-[:90]C(-[:90]H)(-[:0]H)(-[:180]H))(-[:0]H)(-[:180]H)))(-[:270]H)(-[:0]C(-[:90]H)(-[:0]H)(-[:270]H)))))}} \\ \\ \emph{Answer:} 3methyl-hexane, as there is a methyl group on the 3rd carbon of the hexane.}
			\end{enumerate}
			
\end{document}
